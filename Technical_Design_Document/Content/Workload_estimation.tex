\section{Workload Estimation}
Being a tabletop-like game we can look at the stats for tabletop or similar games on PC and check their average and peak players stats. 
We can't take a game like AmongUs as reference, despite the similarities, since it's unrelaistic to expect a similar explosion in popularity. But we can take still popular turn based multiplayer games and treat them as an upper bound.

RISK: Global Domination has an average of 1600-1700 players connected at the same time, with highly variable peaks, between 3000 and 10000+ players.

\fullwidthgraphicscaption{../Pictures/Workload/AmongUs_players_chart.png}{AmongUs recent daily players chart}
\fullwidthgraphicscaption{../Pictures/Workload/AmongUs_players_chart_yearly.png}{AmongUs yearly players chart}
\fullwidthgraphicscaption{../Pictures/Workload/Risk_players_chart.png}{RISK: Global Domination recent daily players chart}
\fullwidthgraphicscaption{../Pictures/Workload/Risk_players_chart_yearly.png}{RISK: Global Domination yearly players chart}

The turn-based nature of the game makes it use very little resources; the network is only used intensively to prevent cheating in some minigames, which are treated as single-player experiences and need no further synchronization with other clients.
Aside for minigames, networking operations and computation are minimal on both the client and the server side.
The servers will have to keep track of currency, highscores and end-game statistics, and purchased cosmetics for each player. Ultimately it's again a negligible amount of data per each player. 50KB should suffice with enough overhead for future content.

After the game is released, besides bugfixing, there will be regular updates introducing new cosmetics, and potentially larger updates introducing new game modes and adding more rooms, minigames and statuses to the game. After the initial peak, we think the interest in the game will stabilize in few weeks, and remain relevant for 2 or at best 3 years.

\pagebreak